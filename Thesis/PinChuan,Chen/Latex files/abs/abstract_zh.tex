\begin{abstractzh}

本研究探討了圖像生成領域中的一項重要問題,即在僅擁有舊元件瑕疵樣本的情況下,如何設計一種能夠有效生成瑕疵特徵的模型結構。焊接作為一種廣泛應用於工業製造的技術,由於高溫、高壓和可能引發化學反應的特點,容易產生缺陷,進而導致製造過程中斷或最終產品品質不佳。工業圖像異常檢測方面已得到廣泛應用。然而,當處理對新類別的圖像檢測時,這些模型通常難以達到滿足工業生產需求的準確度水平。傳統的重新訓練模型方法,以適應每個新類別的樣本,帶來了巨大的人力和計算成本,使得新類別樣本的檢測成為該領域面臨的一個重大挑戰。為解決這一問題,我們提出了一種基於Conditional diffusion model的方法,該模型能夠生成具有特定特徵的圖像,同時彌補新元件瑕疵樣本不足的問題。

在背景說明和相關研究部分,我們回顧了生成模型的發展歷程,從Non-equilibrium thermodynamics到Denoising Diffusion Probabilistic Models(DDPM),最終聚焦於Conditional Diffusion Model。這種模型的特點在於能夠根據不同條件生成具有特定特徵的圖像,並在處理新元件瑕疵樣本不足方面具有優勢。

在研究方法及步驟中,我們提出了一種使用Compositional Conditional Diffusion Model的方法,並詳細描述了模型的訓練過程,包括元件標籤、Embedding生成、Spatial Transformer的應用等。實驗結果顯示,該模型成功生成了未見過的瑕疵新元件,並進行了模型的優化和改進。



\vspace{6cm}

關鍵詞: 深度學習、擴散模型、瑕疵檢測、圖像生成、組合零樣本學習。

\end{abstractzh}